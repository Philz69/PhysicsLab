\documentclass{report}
\usepackage{mathtools}
\usepackage{etoolbox}
\usepackage{amsmath}
\usepackage{siunitx}
\usepackage{tikz}
\usetikzlibrary{datavisualization}
\usetikzlibrary{datavisualization.formats.functions}
\usepackage{titlesec}[rubberchapters]
\usepackage{enumitem}
\usepackage{pgfplots}
\makeatletter
\patchcmd{\chapter}{\clearpage}{}{}{}
\makeatother
\everymath{\displaystyle}
\titleformat{\chapter}[display]
{\normalfont\huge\bfseries}{}{0pt}{\huge}
\titlespacing*{\chapter}{0pt}{0pt}{20pt}
\pgfplotsset{width=10cm,compat=1.9}
\begin{document}
%\begin{titlepage}
%    \centering
%    203-NYB-05\\
%    Electricity and Magnetism\\
%    Professor: Ernest Dubeau\par
%    \vspace{5cm}
%    \Large AC Circuits: Lab 1\par
%    \normalsize
%    By:\\
%    Philipe Goulet\\
%    \vspace*{\fill}
%    {\today}\\
%    Departement of Physics\\
%    Champlain Regional College\\
%    Lennoxville, Quebec, Canada
%\end{titlepage}
\section*{Theory}

While $g$ ( acceleation of gravity ) is often considered as a constant, it changes with altitude. This is because the force of gravity that acts upon any two objects is given by Newton’s law of universal gravitation:
\begin{gather}
    F=G\frac{m_1m_2}{r}
\end{gather}
Where $G$ is the gravitational constant, $r$ is the distance between center of mass of both objects, $m1$ is the mass of object 1 and $m2$ is the mass of object 2\\

However, because earth is not a perfect sphere, r varies depending on altitude. This means that $g$ as mesaured in J-18 will be slightly different from the given value of $9.8 N/kg$. Since measuring the distance between the center of mass and the earth precisely is not something that can be easily done, we have to rely on another method.\\ 

Since the period of a pendulum $T$ only depend on its length $L$  and $g$, we can use it to find $g$ at our location.

\begin{gather}
    T=2\pi\sqrt{\frac{L}{g}}
\end{gather}

\noindent\textcolor{red}{\rule{10cm}{1mm}}

We will be using a Kater’s pendulum because ( I THINK IT IS BECAUSE IT ALLOWS MORE PRECISE MEASUREMENTS ). It is a pendulum with two masses that can be shifted along the length of the pendulum and two fixed pivot points.

When oscillating around Point A, the torque that pushes the pendulum back to its original position is given by:

\begin{equation}\tau=|\vec{ R } x \vec{ F }| = Mgasin\left(\theta\right)\end{equation}

Since clockwise rotation is a negative torque by convention, the restoring force is a negative toque.

\begin{equation}
    \tau=-Mgasin\left(\theta\right)
\end{equation}

Torque is also given by 

\begin{gather*}
    \tau=I\alpha \\
    \alpha=\frac{d\omega}{dt}=\frac{d}{dt}\frac{d\theta}{dt}=\frac{d^2\theta}{dt^2}\\
    \tau=\frac{d^2\theta}{dt^2}
\end{gather*}

Paralel axis theorm bla bla

\begin{equation}
    I_a=I_cm + Ma^2 = M(k^2+a^2)
\end{equation}

Where $I_cm = Mk^2$,$k$ defines the radius of gyration about the center of mass. Using Newton's second law, the second order differential equation for the angular displacement can be written as:

\begin{equation}
    I_a\frac{d^2\theta}{dt^2}=-Mgasin\left(\theta\right) \label{eq:angulardisp}
\end{equation}

%τ=|R x F| = Mgasin(theta)

\noindent\textcolor{red}{\rule{10cm}{1mm}}

By solving Eq. (\ref{eq:angulardisp}) for $T$ ( in this case the period of oscillation with pivot point a ) give us:

\begin{align}
    T_a&=2\pi \sqrt{\frac{I_a}{Mga}}\nonumber\\
    T_a&=2\pi \sqrt{\frac{M(k^2+a^2)}{ Mga }}\nonumber\\
    T_a&=2\pi \sqrt{\frac{k^2+a^2}{ga}} \label{eq:perioda}
\end{align}
Doing the same with pivot point B gives us:
\begin{gather}
    T_b=2\pi \sqrt{\frac{k^2+b^2}{gb}}\label{eq:periodb}
\end{gather}

When $T_a = T_b$:  
\begin{align*}
    2\pi \sqrt{\frac{k^2+a^2}{ga}}&=2\pi \sqrt{\frac{k^2+b^2}{gb}}\\
    \frac{k^2+a^2}{ ga }&= \frac{ k^2+b^2 }{ ga }\\
    \frac{ k^2+a^2 }{ ga }&= \frac{ k^2+b^2 }{ gb }\\
    \frac{ k^2+a^2 }{ a }&= \frac{ k^2+b^2 }{ b }\\
    (k^2+a^2)b &=(k^2+b^2)a \\
    (k^2+a^2)b-(k^2+b^2)a &= 0\\
    a^2b-(k^2+b^2)a + k2b &=0
\end{align*}

The above is an equation of the form $ax^2+bx+c$, thus we apply the quadratic formula
\begin{gather*}
    a= \frac{\left(k^2+b^2\right)\pm \sqrt{\left(-\left(k^2+b^2\right)\right)^2-4bk^2b}}{2b}\\
    a= \frac{\left(k^2+b^2\right)\pm \left(k+b\right)\left(k-b\right)}{ 2b }
\end{gather*}
\begin{equation}
    \begin{split}
        a_1 &= \frac{ (k^2+b^2)+ (k+b)(k-b) }{2b}\\
        a_1 &= \frac{ k^2+b^2+k2+kb-kb-b^2 }{2b} \\
        a_1 &= \frac{2k^2}{2b}\\
        a_1 &= \frac{k^2}{b}\\
        ab &= k^2 \label{eq:solve1} 
    \end{split}
    \qquad
    \begin{split}
        a_2 &= \frac{ (k^2+b^2)- (k+b)(k-b) }{ 2b } \\
        a_2 &= \frac{ k^2+b^2-k2-kb+kb+b^2 }{ 2b }   \\
        a_2 &= \frac{ 2b^2} { 2b }                    \\
        a_2 &= \frac{b^2}{ b }                      \\
        a &=b                          
    \end{split}
\end{equation}

By using $ab=k^2$ obtained in Eq. (\ref{eq:solve1}) to replace $k^2$ in \ref{eq:perioda} and \ref{eq:periodb} 

\begin{equation}
    \begin{aligned}
        T_a&=T_b \\
        2\pi \sqrt{\frac{k^2+a^2}{ga}} &=2\pi \sqrt{\frac{k^2+b^2}{gb}} \\
        2\pi \sqrt{\frac{ab+a^2}{ga}} &=2\pi \sqrt{\frac{ab+b^2}{gb}} \\
        2\pi \sqrt{\frac{a\left(a+b\right)}{ga}} &=2\pi \sqrt{\frac{b\left(a+b\right)}{gb}} \\
        2\pi \sqrt{\frac{\left(a+b\right)}{g}} &=2\pi \sqrt{\frac{\left(a+b\right)}{g}} \\
        T&=2\pi \sqrt{\frac{a+b}{ g }} \label{eq:periodab}
    \end{aligned}
\end{equation}


We then solve the Eq. (\ref{eq:periodab}) for $g$

\begin{equation}
    \begin{aligned}
        T&=2\pi \sqrt{\frac{a+b}{ g }} \\
        T^2&=4\pi^2 \frac{a+b}{ g } \\
        g&=4\pi^2\frac{ a+b }{ T^2 }
    \end{aligned}
\end{equation}

$a+b$ being the distance between the fixed pivot points, which is given in this case as being $a+b=100.06cm$
Using the final derived equation, we will be able to easily calculate g at our location just by setting up the masses on the pendulum in such a way that Ta and Tb are equal and then using that period in the equation



\end{document}
